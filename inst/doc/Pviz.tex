%\VignetteIndexEntry{The Pviz users guide}
%\VignetteDepends{Pviz}
%\VignetteKeywords{Visualization}
%\VignettePackage{Pviz}
\documentclass[11pt]{article}
\usepackage{Sweave}
\usepackage{underscore}
\usepackage[authoryear,round]{natbib}


\author{Renan Sauteraud\footnote{rsautera@fhcrc.org}}

\begin{document}
\title{The Pviz User Guide}
\maketitle

\tableofcontents

\newpage

\section{Example of plot}
\begin{Schunk}
\begin{Sinput}
> library(HIV.db)
> library(Pviz)
> ################## Get reference scale from alignment
> alignObj<-readAlign()
> refScale<-alignObj[[1]]
> refSeq<-alignObj[[2]]
> ################## Load database
> HIV_db<-loadFeatures(ref="env")
> envBase<-getFeature(HIV_db)
> envStart=getHXB2Coordinates(envBase)[1,][1] 
> envEnd=getHXB2Coordinates(envBase)[1,][2] 
> ################## Getting Features from database for annotation
> proteins<-getFeature(HIV_db,category="protein",start=envStart,end=envEnd,frame=getFrame(envBase))
> antis<-getEpitope(envBase,name=c("VRC01"))
> helix<-getChildren(envBase,category=c("helix"))
\end{Sinput}
\end{Schunk}

\begin{Schunk}
\begin{Sinput}
> ## ProteinAxisTrack using the extended coordinates system
> rpext<-ProteinAxisTrack(littleTicks=TRUE)
> ## ProteinAxisTrack with coordinates relative to the reference
> rpref<-ProteinAxisTrack(refScale=refScale, adNC=TRUE)
\end{Sinput}
\end{Schunk}

\begin{Schunk}
\begin{Sinput}
> sTrack<-SequenceTrack(refSeq)
\end{Sinput}
\end{Schunk}

\begin{Schunk}
\begin{Sinput}
> data(pepMicroarrayEx)
> p1Track<-ProbeTrack(pepMicroarrayEx$probeSeq, pepMicroarrayEx$probeFreq, pepMicroarrayEx$probePos
+ 		, protein="gp120", name="sequence(B)")
\end{Sinput}
\end{Schunk}

\begin{Schunk}
\begin{Sinput}
> a2Track<-ATrack(id=proteins@values@unlistData@listData[["name"]],start=start(proteins),end=end(proteins),genome='hxb2',name="Protein",protein="gp120",fill="navyblue",size=1)
> a3Track<-ATrack(id=helix@values@unlistData@listData[["name"]],start=start(helix),end=end(helix),genome='hxb2',name="Helix",fill="orange",protein="gp120")
> a6Track<-ATrack(id=antis@values@unlistData@listData[["name"]],start=start(antis),end=end(antis),genome='hxb2',name="Epitopes",fill="gray",protein="gp120")
\end{Sinput}
\end{Schunk}

\begin{Schunk}
\begin{Sinput}
> data(pepExprEx)
> library(IRanges)
> d6Track<-DTrack(range=IRanges(start=pepExprEx$dPos,width=1),groups=rownames(pepExprEx$dExpr),data=pepExprEx$dExpr,genome='hxb2',protein="gp120",col=c("orange","gray"),cex=1)
\end{Sinput}
\end{Schunk}

\begin{Schunk}
\begin{Sinput}
> plotTracks(trackList=c(rpext,rpref,sTrack,a2Track,a3Track,a6Track,p1Track,d6Track), from=1, to=150, type=c("p","smooth"), stacking="dense",legend=TRUE, showFeatureId=TRUE)
\end{Sinput}
\end{Schunk}

\section{Using the extended coordinates system}

The extended coordinate system is based on a multiple alignment. It is a scale from 0 to the length of the alignmenat (i.e: reference sequence + gaps).

The refScale is a scale used to translate coordinates between the extended system and the normal system (which is based uniquely on the reference sequence length).

\subsection{Data formating}
In the following examples, hxb2 is the reference sequence and is aligned with different subtypes.

First, get the refScale:
\begin{Schunk}
\begin{Sinput}
> alignObj<-readAlign(filename=system.file("extdata/alignment.fasta", package="Pviz"))
> refScale<-alignObj[[1]]
> refSeq<-alignObj[[2]]
\end{Sinput}
\end{Schunk}

Now, the refScale can be used to translate coordiantes into the extended system.

Example of loading HIV_db using extended coordinates. 
\begin{Schunk}
\begin{Sinput}
> library(HIV.db)
> HIV_db<-loadFeatures(ref="env", refScale=refScale)
> envBase<-getFeature(HIV_db)
\end{Sinput}
\end{Schunk}

The positions in envBase are the ones observed in the alignment.

To translate all the positions in pep_hxb2, use convertPep().
\begin{Schunk}
\begin{Sinput}
> data(pep_hxb2)
> nrd<-convertPep(rd=pep_hxb2)
\end{Sinput}
\end{Schunk}

nrd is pep_hxb2 with updated ranges, aligned and trimmed columns.

To convert coordinates into the extended system, use coord2ext() with the refScale defined earlier:
\begin{Schunk}
\begin{Sinput}
> start<-coord2ext(c(200,450), ref=refScale)
> start
\end{Sinput}
\begin{Soutput}
[1] 223 474
\end{Soutput}
\end{Schunk}

\subsection{Plotting with both coordinates system}
All objects used to create tracks should be in extended coordinate system.
\begin{Schunk}
\begin{Sinput}
> sTrack<-SequenceTrack(refSeq)
> pax1<-ProteinAxisTrack()
> pax2<-ProteinAxisTrack(refScale=refScale, col.gap="blue")
> proteins<-getFeature(HIV_db,category="protein",start=envStart,end=envEnd,frame=getFrame(envBase))
> aTrack<-ATrack(id=proteins@values@unlistData@listData$name,start=start(proteins),end=end(proteins))
> aTrack<-ATrack(id=helix@values@unlistData@listData$name,start=start(helix),end=end(helix),fill="red")
> plotTracks(c(pax1,pax2,sTrack,aTrack), from=1, to=160)
\end{Sinput}
\end{Schunk}
The first axis track displays the extended scale, while the second displays the reference coordinates, it also shows the gaps in the reference sequence with respect of the alignment.

\end{document}
